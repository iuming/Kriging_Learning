\section{克里金插值原理}
Kriging方法是对于连续空间变化的随机模型,通过稀疏样本数据来预测未测量站点值的一种方法,其插值由先验协方差控制的高斯过程建模。在先验的适当假设下,克里金法可为中间值提供最佳的线性无偏预测。普通克里格法不需要过多额外信息,只需要这些信息加上测量值和它们的地理坐标。这是目前为止最流行的克里金方法;它在大多数情况下都能够应用,因为它的应用条件很容易得到。

普通克里金方法是基于这样的假设:连续空间变化是随机的并且与空间相关。由于随机过程具有恒定的平均值与方差,因此预测点的值取决于相对位置,而与绝对位置无关。于是,我们把目标值分为确定性趋势值以及随机的自相关函数值:
\begin{equation}
    V\left( \mathbf{s} \right) = \mu\left( \mathbf{s} \right) + \varepsilon\left( \mathbf{s} \right)
    \label{目标值分为确定性数值和随机自相关数值}
\end{equation}
其中$ V\left( \mathbf{s} \right) $是目标值,$ \mu\left( \mathbf{s} \right) $是确定性趋势值,$ \varepsilon\left( \mathbf{s} \right) $是随机的自相关函数值,$ \mathbf{s} $为标识位置,在三维空间中可视为空间$ x $(经度)、$ y $(纬度)、$ z $(高度)坐标。

假设有$ N $个随机变量$ V $的值已经记录在采样点$ \mathbf{s}_{1}, \mathbf{s}_{2}, \mathbf{s}_{3} , \cdots , \mathbf{s}_{N} $作为已知点,对于克里金插值,我们通过以下公式预测其插值点$ \mathbf{s}_{0} $的值:
\begin{equation}
    \hat{V}\left( \mathbf{s}_{0} \right) = \sum_{i=1}^{N} \lambda_{i} V\left( \mathbf{s}_{i} \right)
    \label{点克里金插值公式}
\end{equation}
其中,$ \lambda_{i} $为权重。为了确保估计值无偏以及权重之和为1,有:
\begin{equation}
    E\left[ \hat{V}\left( \mathbf{s}_{0} \right) - V\left( \mathbf{s}_{0} \right) \right] = 0
    \label{点克里金插值无偏}
\end{equation}
\begin{equation}
    \sum_{i=1}^{N} \lambda_{i} = 1
    \label{点克里金插值权重之和为1}
\end{equation}

预测方差为:
\begin{equation}
    \begin{split}
        Var\left[ \hat{V} \left( \mathbf{s}_{0} \right) \right]
        & = E\left[ {\hat{V}\left( \mathbf{s}_{0} \right) - V\left( \mathbf{s}_{0} \right)}^{2} \right]     \\
        & = 2 \sum_{i=1}^{N} \lambda_{i} \gamma\left( \mathbf{s}_{i} - \mathbf{s}_{0} \right) - \sum_{i=1}^{N} \sum_{i=1}^{N} \lambda_{i} \lambda_{j} \gamma\left( \mathbf{s}_{i} - \mathbf{s}_{j} \right)
    \end{split}
    \label{点克里金插值预测方差}
\end{equation}
其中,函数$ \gamma\left( \mathbf{s}_{i} - \mathbf{s}_{0} \right) $代表采样点$ \mathbf{s}_{i} $和目标预测点$ \mathbf{s}_{0} $之间的半方差函数(也称半变异函数);$ \gamma\left( \mathbf{s}_{i} - \mathbf{s}_{j} \right) $是采样点$ \mathbf{s}_{i} $和$ \mathbf{s}_{j} $之间的半方差函数。

半方差函数是从变异函数模型中导出的,一方面是因为没有观测值的数据点和目标点之间没有半方差的度量,另一方面原因是只有这样才能保证方差不是负的。如果一个目标点碰巧也是一个采样点,那么克里金方法直接返回那里的观测值,估计方差为零。

前面所提到的克里金技术都是在预测特定未采样位置的变量值,这些位置可以被视为空间点,因此,这种克里金法也被称为点克里金法。当不确定性相对较大时,可能需要通过比单个点更大的区域上执行克里金法来得到平滑插值结果,这样的方法称为块克里金法。相比于点克里金法,块克里金法具有降低空间预测误差的优势,但是它可能丢失一些有用信息。但是为了得到更好的插值重构效果,块克里金表现的更好。

与点克里金法插值公式相似,块克里金法的任意块估计值也是测量值的加权平均值:
\begin{equation}
    \hat{V}\left( B \right) = \sum_{i=1}^{N} \lambda_{i} V \left( \mathbf{s}_{i} \right)
    \label{块克里金插值公式}
\end{equation}

与点克里金一样,块克里金公式中的$ \lambda_{i} $之和为1,其预测值方差为:
\begin{equation}
    \begin{split}
        Var\left[ \hat{V} \left( \mathbf{B} \right) \right]
        & = E\left[ {\hat{V}\left( \mathbf{B} \right) - V\left( \mathbf{B} \right)}^{2} \right]     \\
        & = 2 \sum_{i=1}^{N} \lambda_{i} \gamma\left( \mathbf{s}_{i} , \mathbf{B} \right) - \sum_{i=1}^{N} \sum_{i=1}^{N} \lambda_{i} \lambda_{j} \gamma\left( \mathbf{s}_{i} - \mathbf{s}_{j} \right) - \overline{\gamma} \left( \mathbf{B}, \mathbf{B} \right)
    \end{split}
    \label{块克里金插值预测方差}
\end{equation}
其中,$ \gamma\left( \mathbf{s}_{i} , \mathbf{B} \right) $是数据点$ \mathbf{s}_{i} $和目标块$ \mathbf{B} $之间的平均半方差函数;$ \overline{\gamma} \left( \mathbf{B}, \mathbf{B} \right) $是目标块$ \mathbf{B} $内的平均半方差,即块内方差。

下一步,要在权重之和为1的约束下,寻找使得预测值方差最小的权重。通过公式\ref{点克里金插值公式}-\ref{点克里金插值预测方差},可推导出$ N+1 $个公式,其中包含$ N+1 $个未知数:
\begin{equation}
    \sum_{i=1}^{N} \lambda_{i} \gamma \left( \mathbf{s}_{i} - \mathbf{s}_{j} \right) + \psi \left( \mathbf{s}_{0} \right) = \gamma \left( \mathbf{s}_{j} - \mathbf{s}_{0} \right) \qquad \text{for all $ j $}
    \label{点克里金法N个方程}
\end{equation}
\begin{equation}
    \sum_{i=0}^{N}\lambda_{i} = 1
    \label{点克里金法N+1个方程}
\end{equation}
其中,$ \psi \left( \mathbf{s}_{0} \right) $为拉格朗日数乘。

展开作为方程组形式为:
\begin{equation}
    \left\{
    \begin{aligned}
        \gamma \left( \mathbf{s}_{1} - \mathbf{s}_{1} \right) \lambda_{1}+ \gamma \left( \mathbf{s}_{1} - \mathbf{s}_{2} \right) \lambda_{2}+\cdots + \gamma \left( \mathbf{s}_{1} - \mathbf{s}_{n} \right) \lambda_{n}-\psi\left( \mathbf{s}_{0} \right)    & = \gamma \left( \mathbf{s}_{1} - \mathbf{s}_{0} \right) \\
        \gamma \left( \mathbf{s}_{2} - \mathbf{s}_{1} \right) \lambda_{1} + \gamma \left( \mathbf{s}_{2} - \mathbf{s}_{2} \right) \lambda_{2} + \cdots + \gamma \left( \mathbf{s}_{2} - \mathbf{s}_{n} \right) \lambda_{n}-\psi\left( \mathbf{s}_{0} \right) & = \gamma \left( \mathbf{s}_{2} - \mathbf{s}_{0} \right) \\                                                                & \cdots                                                  \\
        \gamma \left( \mathbf{s}_{n} - \mathbf{s}_{1} \right) \lambda_{1}+\gamma \left( \mathbf{s}_{n} - \mathbf{s}_{2} \right) \lambda_{2}+\cdots+\gamma \left( \mathbf{s}_{n} - \mathbf{s}_{n} \right)\lambda_{n}-\psi\left( \mathbf{s}_{0} \right)        & = \gamma \left( \mathbf{s}_{n} - \mathbf{s}_{0} \right) \\
        \lambda_{1}+\lambda_{2}+\cdots+\lambda_{n}                                                                                                                                                                                                           & =1
    \end{aligned}
    \right.
    \label{克里金方程线性方程组}
\end{equation}

由克里金方程中每个测量点所占的权重,可将预测方差公式变为:
\begin{equation}
    \sigma^{2}\left( \mathbf{s}_{0} \right) = \sum_{i=1}^{N} \lambda_{i} \gamma \left( \mathbf{s}_{i} - \mathbf{s}_{0} \right) + \psi \left( \mathbf{s}_{0} \right)
    \label{点克里金法预测方差公式}
\end{equation}

克里金方程也可以写成矩阵形式:
\begin{equation}
    \mathbf{A} \mathbf{\lambda} = \mathbf{b}
    \label{点克里金方程矩阵形式}
\end{equation}
其中,矩阵$ \mathbf{A} $表示第$ i $个取样点和第$ j $个取样点之间的半方差;$ \mathbf{\lambda} $表示权重和拉格朗日乘子向量;$ \mathbf{b} $表示每个采样点和目标点之间的半方差向量。

矩阵的具体形式为:
\begin{equation}
    \left[\begin{array}{ccccc}
            \gamma \left( \mathbf{s}_{1} - \mathbf{s}_{1} \right) & \gamma \left( \mathbf{s}_{1} - \mathbf{s}_{2} \right) & \cdots & \gamma \left( \mathbf{s}_{1} - \mathbf{s}_{n} \right) & 1      \\
            \gamma \left( \mathbf{s}_{2} - \mathbf{s}_{1} \right) & \gamma \left( \mathbf{s}_{2} - \mathbf{s}_{2} \right) & \cdots & \gamma \left( \mathbf{s}_{2} - \mathbf{s}_{n} \right) & 1      \\
            \cdots                                                & \cdots                                                & \cdots & \cdots                                                & \cdots \\
            \gamma \left( \mathbf{s}_{n} - \mathbf{s}_{1} \right) & \gamma \left( \mathbf{s}_{n} - \mathbf{s}_{2} \right) & \cdots & \gamma \left( \mathbf{s}_{n} - \mathbf{s}_{n} \right) & 1      \\
            1                                                     & 1                                                     & \cdots & 1                                                     & 0
        \end{array}\right]\left[\begin{array}{c}
            \lambda_{1} \\
            \lambda_{2} \\
            \cdots      \\
            \lambda_{n} \\
            -\psi\left( \mathbf{s}_{0} \right)
        \end{array}\right]=\left[\begin{array}{c}
            \gamma \left( \mathbf{s}_{1} - \mathbf{s}_{0} \right) \\
            \gamma \left( \mathbf{s}_{2} - \mathbf{s}_{0} \right) \\
            \ldots                                                \\
            \gamma \left( \mathbf{s}_{n} - \mathbf{s}_{0} \right) \\
            1
        \end{array}\right]
\end{equation}

矩阵$ \mathbf{A} $可逆,权重和拉格朗日乘子向量可表示为:
\begin{equation}
    \mathbf{\lambda} = \mathbf{A}^{-1} \mathbf{b}
    \label{点克里金方程逆矩阵计算权重}
\end{equation}

矩阵形式计算预测方差为:
\begin{equation}
    \hat{\sigma}^{2} \left( \mathbf{s}_{0} \right) = \mathbf{b}^{T} \lambda
    \label{点克里金方程矩阵形式计算预测方差}
\end{equation}

相同的,块克里金法也可按照以上推导,可得出计算公式:
\begin{equation}
    \sum_{i=1}^{N} \lambda_{i} \gamma \left( \mathbf{s}_{i} - \mathbf{s}_{j} \right) + \psi \left( \mathbf{B} \right) = \overline{\gamma} \left( \mathbf{s}_{j} , \mathbf{B} \right) \qquad \text{for all $ j $}
\end{equation}
\begin{equation}
    \sum_{i=1}^{N} \lambda_{i} = 1
\end{equation}

块克里金预测方差为:
\begin{equation}
    \sigma^{2}\left( \mathbf{B} \right) = \sum_{i=1}^{N} \lambda_{i} \overline{\gamma} \left( \mathbf{s}_{i} , \mathbf{B} \right) + \psi \left( \mathbf{B} \right) - \overline{\gamma} \left( \mathbf{B} , \mathbf{B} \right)
\end{equation}

在块克里金法的矩阵表示中,$ \mathbf{b} $表示每个采样点和目标块之间的半方差向量,块克里金预测方差可表示为:
\begin{equation}
    \hat{\sigma}^{2} \left( \mathbf{B} \right) = \mathbf{b}^{T} \mathbf{\lambda} - \hat{\gamma} \left( \mathbf{B} , \mathbf{B} \right)
\end{equation}

通常情况下,块克里金法的预测方差都小于点克里金法,因为任何块克里金方差都完全包含在块内方差中。块克里金预测之间的波动也比点克里金预测之间的波动小一些,因此块克里金法生成的辐射场要比点克里金法生成的辐射场更加平滑。

上面描述的克里金方程中充分的表现出了区域化变量理论以及变异函数与模型的重要性,因此,接下来的部分仔细讲述区域化变量理论和变异函数与模型的内容。

\section{区域化变量理论}
平稳性是克里金方法实用性的基础,假设插值区域数据视为有一定趋势的确定性变化,可以将插值空间场的数值分为两个部分,即确定性趋势函数值和随机自相关函数值:
\begin{equation}
    V \left( \mathbf{s} \right) = \mu \left( \mathbf{s} \right) + \varepsilon \left( \mathbf{s} \right)
    \label{区域化变量}
\end{equation}
其中,$ \mu \left( \mathbf{s} \right) $为确定性函数变量;$ \varepsilon \left( \mathbf{s} \right) $为均值为0,协方差为$ C\left( \mathbf{h} \right) $的随机自相关变量,其中$ \mathbf{h} $是空间中的相对距离,称为滞后。

协方差表达式为:
\begin{equation}
    C\left( \mathbf{h} \right) = E \left[ \varepsilon \left( \mathbf{s} \right) \varepsilon \left( \mathbf{s} + \mathbf{h} \right) \right]
    \label{协方差表达式}
\end{equation}

当$ \mu \left( \mathbf{s} \right) $为常数时(也就是普通克里金方法,$ \mu \left( \mathbf{s} \right) = \mu \left( \text{常数} \right) $时),协方差函数可以表达为:
\begin{equation}
    C\left( \mathbf{h} \right) = E \left[ \{ V\left( \mathbf{s} \right) - \mu \} \{ V\left( \mathbf{s} + \mathbf{h} \right) - \mu \} \right] = E \left[ \{ V \left( \mathbf{s} \right) \} \{ V \left( \mathbf{s} + \mathbf{h} \right) \} - \mu^{2} \right]
    \label{协方差函数}
\end{equation}
其中,$  V \left( \mathbf{s} \right) $和$ V \left( \mathbf{s} + \mathbf{h} \right) $表示空间场物理变量在$ \mathbf{s} $和$ \mathbf{s} + \mathbf{h} $的值,$ E $表示期望。

从公式\ref{协方差函数}看出,协方差仅与$ \mathbf{h} $有关,即样本之间在距离和方向上的间隔。这是基于二阶平稳性的假设,在实际空间场插值过程中,通常不能假设平均值是常数,否则方差就不存在。实际情况中,通常将平稳性假设改为Matheron所提出的内在平稳性假设\textsuperscript{\cite{matheron1965variables}},即期望差为0:
\begin{equation}
    E\left[ V\left( \mathbf{s} \right) - V \left( \mathbf{s} + \mathbf{h} \right) \right] = 0
\end{equation}

从而可以推导出变异函数与预测值方差的关系:
\begin{equation}
    Var \left[ V\left( \mathbf{s} \right) - V\left( \mathbf{s} + \mathbf{h} \right) \right] = E\left[ \{ V\left( \mathbf{s} \right) - V \left( \mathbf{s} + \mathbf{h} \right) \}^{2} \right] = 2 \gamma\left( \mathbf{h} \right)
\end{equation}
其中,$ \gamma\left( \mathbf{h} \right) $是滞后$ \mathbf{h} $的半方差,它是与相对位置$ \mathbf{h} $有关的变异函数。

对于二阶平稳状态,协方差函数和变异函数是等价的:
\begin{equation}
    \gamma \left( \mathbf{h} \right) = C \left( \mathbf{0} \right) - C \left( \mathbf{h} \right)
\end{equation}
其中,$ C \left( \mathbf{0} \right) = \sigma^{2} $表示方差。

因此,对于普通克里金模型,半方差函数可以看成:
\begin{equation}
    \begin{split}
        \gamma\left( \mathbf{h} \right)
        & = \frac{1}{2} E \left[ \{ \varepsilon\left( \mathbf{s} \right) - \varepsilon \left( \mathbf{s} + \mathbf{h} \right)\}^{2} \right] \\
        & = \frac{1}{2} E \left[ \{ V \left( \mathbf{s} \right) - V \left( \mathbf{s} + \mathbf{h} \right)\}^{2} \right]
    \end{split}
\end{equation}

\section{变异函数与模型}
变异函数是许多地质统计学应用的基石。在统计学应用中,变异函数和任何与之适应的模型都应该是准确的,只有这样,重构出的数据才能更好地符合实际。克里金法需要使用一个变异函数,来得到最小地克里金预测方差。

\subsection{半方差有序集}
获得变异函数地第一步是通过已知的测量点数据$ V\left( \mathbf{s}_{1} \right) , V\left( \mathbf{s}_{2} \right) , \cdots $,(其中$ \mathbf{s}_{1} , \mathbf{s}_{2} , \cdots $表示样本在三维空间中的位置)来估算变异函数。在估算变异函数之前,需要保证选取的这些样本点是随机的,因为在定义变异函数时我们认为变量为随机过程的结果。

估算变异函数的常用公式有Matheron矩量法估计(MoM):
\begin{equation}
    \hat{\gamma}\left( \mathbf{h} \right) = \frac{1}{2m\left( \mathbf{h} \right)} \sum_{i=1}^{m\left( \mathbf{h} \right)} \{v\left( \mathbf{s}_{i} \right) - v\left( \mathbf{s}_{i} + \mathbf{h} \right)\}^{2}
    \label{Matheron矩量法估计}
\end{equation}
其中,$ v\left( \mathbf{s}_{i} \right) $和$ v\left( \mathbf{s}_{i} + \mathbf{h} \right) $为位置在$ \mathbf{s}_{i} $和$ \mathbf{s}_{i} + \mathbf{h} $的测量值;$ m\left( \mathbf{h} \right) $为在滞后$ \mathbf{h} $处的成对比较次数。

通过改变$ \mathbf{h} $,可以得到半方差的有序集,这些构成估计变异函数的数据。

\subsection{变异函数模型}
变异函数模型主要可以分为两类——有界函数和无界函数。最常用的三个变异函数模型为:幂函数(无界)、球函数(有界)和指数函数(渐进有界)。如果常用的变异函数不能够符合实际值,也可以拟合更加复杂的函数。

下面介绍四种常见的变异函数:
\paragraph{指数函数}
指数函数是一种无界函数:
\begin{equation}
    \gamma\left( h \right) = c_{0} + g h^{\beta} \qquad \left( 0 < \beta < 2 \right)
\end{equation}
其中,$ c_{0} $为纵坐标截距,$ g $为变化的程度,$ \beta $为指数幂。
\paragraph{球函数}
球函数是一种有界函数(分段函数):
\begin{equation}
    \gamma\left( h \right) =
    \left\{
    \begin{aligned}
         & c_{0} + c\{ \frac{3h}{2r} - \frac{1}{2} \left( \frac{h}{r} \right)^{3} \}    \qquad & \left( 0 < h \leq r \right) \\
         & c_{0} + c                                                                           & \left( h > r \right)        \\
         & c                                                                                   & \left( h = 0 \right)
    \end{aligned}
    \right.
\end{equation}
其中,$ c_{0} $为块金方差;$ c $为空间相关程度的方差;$ r $为空间范围。
\paragraph{指数函数}
指数函数是一种渐进有界函数:
\begin{equation}
    \gamma \left( h \right) =
    \left\{
    \begin{aligned}
         & c_{0} + c \{ 1 - \exp \left( - \frac{h}{a} \right) \} \qquad & \left( 0 < h \right) \\
         & 0                                                            & \left( h = 0 \right)
    \end{aligned}
    \right.
\end{equation}
其中,$ a $为距离参数。该函数模型渐进地接近平滑,它的定义域为无穷,但是为了接近实际,通常定义一个有效距离范围。
\paragraph{嵌套球形函数}
嵌套球形函数是在不同范围采用不同的球函数:
\begin{equation}
    \gamma \left( h \right) =
    \left\{
    \begin{aligned}
         & c_{0} + c_{1} \{ \frac{3h}{2r_{1}} - \frac{1}{2}\left( \frac{h}{r_{1}} \right)^{3} \} + c_{2} \{ \frac{3h}{2r_{2}} - \frac{1}{2}\left( \frac{h}{r_{2}} \right)^{3} \} \qquad & \left( 0 < h \leq r_{1} \right)     \\
         & c_{0} + c_{1} + c_{2} \{ \frac{3h}{2r_{2}} - \frac{1}{2}\left( \frac{h}{r_{2}} \right)^{3} \}                                                                                & \left( r_{1} < h \leq r_{2} \right) \\
         & c_{0} + c_{1} + c_{2}                                                                                                                                                        & \left( h > r_{2} \right)            \\
         & 0                                                                                                                                                                            & \left( h = 0 \right)
    \end{aligned}
    \right.
\end{equation}
其中,$ c_{1},r_{1} $为近距离分量的空间相关程度方差和范围;$ c_{2},r_{2} $为远距离分量的空间相关程度方差和范围。

\subsection{选取变异函数}
在地质统计学中,变异函数拟合模型仍然存在争议,尽管它是克里金插值最重要的步骤之一。一些工程师们凭借肉眼来选取拟合模型,这可能导致半方差在点与点之间波动过大,并且它的准确性是不稳定的。因此,常常会有些选取指标来辅助我们来选取变异函数模型,例如RSS(残差平方和)、AIC(Akaike信息准则)。选取变异函数模型通常经过以下几个步骤:
\begin{enumerate}
    \item 将公式\ref{Matheron矩量法估计}计算出的半方差有序集,作为散点绘制在坐标图中;
    \item 选择几个形状相似的模型,用加权最小二乘法依次进行拟合,绘制拟合曲线;
    \item 评估选取的变异函数模型是否合理。若选取的模型都拟合得很好,则选择RSS最小的模型;若模型参数个数不相等,则选择AIC最少的模型。
\end{enumerate}

AIC估计公式为:
\begin{equation}
    AIC = \{ n \ln \left( \frac{2 \pi}{n} \right) + n + 2 \} + n \ln R + 2 p
\end{equation}
其中,$ n $为半方差有序集中点的个数;$ p $为模型参数的个数;$ R $为残差的均方。大括号里的表达式,对于任何选取模型都是常数,因此,判断AIC只需要计算:
\begin{equation}
    \hat{AIC} = n \ln R + 2 p
\end{equation}

\section{克里金法的多变量扩展}
为了解决石油工程、采矿和地质、气象、土壤科学、精确农业、污染控制、公共卫生、渔业、动植物生态、遥感和水文学等方面日益复杂的问题,克里金法已经发展出了许多不同种类的克里金方法。

根据公式\ref{目标值分为确定性数值和随机自相关数值}中确定性函数的不同类型,可以将克里金法分为普通克里金法、泛克里金法、简单克里金法、指示克里金法、协同克里金法、概率克里金法、析取克里金法等。
\paragraph{普通克里金法}
普通克里金法假设的模型为:
\begin{equation}
    V\left( \mathbf{s} \right) = \mu + \varepsilon \left( \mathbf{s} \right)
\end{equation}
其中,$ \mu $为一个未知常量。

选取普通克里金法,最主要的问题之一就是判断常量平均值的假设是否合理。普通克里金法凭借其显著的灵活性,是现如今应用最为广泛的克里金法。
\paragraph{泛克里金法}
泛克里金法的假设模型为:
\begin{equation}
    V\left( \mathbf{s} \right) = \mu\left( \mathbf{s} \right) + \varepsilon\left( \mathbf{s} \right)
\end{equation}
其中,$ \mu\left( \mathbf{s} \right) $为带有未知参量的函数,例如多项式函数、幂指函数等。
\paragraph{简单克里金法}
简单克里金法的假设模型为:
\begin{equation}
    V\left( \mathbf{s} \right) = \mu + \varepsilon\left( \mathbf{s} \right)
\end{equation}
其中,$ \mu $为已知常量,或者没有未知参量的函数。
\paragraph{指示克里金法}
指示克里金法的假设模型为:
\begin{equation}
    I\left( \mathbf{s} \right) = \mu + \varepsilon\left( \mathbf{s} \right)
\end{equation}
其中,$ \mu $为一个未知参量;$ I\left( \mathbf{s} \right) $为一个二进制变量。二进制数据的创建可利用连续数据的阈值实现,或观测数据可以为0或1。指示克里金法可用于离散数据的插值。
\paragraph{协同克里金法}
协同克里金法的假设模型为:
\begin{equation}
    \left\{
    \begin{aligned}
         & V_{1}\left( \mathbf{s} \right) = \mu_{1} + \varepsilon_{1}\left( \mathbf{s} \right) \\
         & V_{2}\left( \mathbf{s} \right) = \mu_{2} + \varepsilon_{2}\left( \mathbf{s} \right)
    \end{aligned}
    \right.
\end{equation}
其中,$ \mu_{1} $和$ \mu_{2} $为未知常量;$ \varepsilon_{1}\left( \mathbf{s} \right) $和$ \varepsilon_{2}\left( \mathbf{s} \right) $为两个随机误差各自的自相关函数,它们之间存在互相关。
\paragraph{概率克里金法}
概率克里金法的假设模型为:
\begin{equation}
    \left\{
    \begin{aligned}
         & I\left( \mathbf{s} \right) = I\left( V\left( \mathbf{s} \right) > c_{t} \right) = \mu_{1} + \varepsilon_{1}\left( \mathbf{s} \right) \\
         & V\left( \mathbf{s} \right) = \mu_{2} + \varepsilon_{2}\left( \mathbf{s} \right)
    \end{aligned}
    \right.
\end{equation}
其中,$ \mu_{1} $和$ \mu_{2} $为未知常量,$ I\left( \mathbf{s} \right) $为通过使用阈值指示$ I\left( V\left( \mathbf{s} \right) > c_{t} \right) $创建的二进制变量。
\paragraph{析取克里金法}
析取克里金法的假设模型为:
\begin{equation}
    f\left( V\left( \mathbf{s} \right) \right) = \mu_{1} + \varepsilon\left( \mathbf{s} \right)
\end{equation}
其中,$ \mu_{1} $为一个未知常量;$ f\left( V\left( \mathbf{s} \right) \right) $为$ V\left( \mathbf{s} \right) $的一个任意函数。若$ f\left( V\left( \mathbf{s} \right) \right) = I\left( V\left( \mathbf{s} \right) > c_{t} \right) $,则指示克里金就是析取克里金的一种特殊情况。