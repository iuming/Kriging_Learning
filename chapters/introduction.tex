\section{介绍}
“Kriging”一词在空间统计中的使用已经成为“最佳预测”的同义词\textsuperscript{\cite{cressie1990origins}},应用于已知测量点附近位置的预测。20世纪50年代,南非金矿开采工程师D.G. Krige提出了几种统计方法来预测平均金矿品位。受他方法的启发,法国数学家马瑟顿(G. Matheron)在20世纪60年代为随机场数据首次提出了一种基于回归的空间预测方法。马瑟顿为感激Krige提出的想法创造Kriging一词,以表彰Krige的开拓性工作\textsuperscript{\cite{lovric2011international}}。

克里金方法是一种地质统计学的预测方法,地质统计学最初是在采矿业中发展起来的,现在广泛应用于环境科学——陆地、大气和海洋\textsuperscript{\cite{oliver2015basic}}。环境科学研究者需要绘制出带有属性(降雨量、土壤营养等)的地图,这些地图上有或多或少的连续区域,但是通常只有稀疏的样本数据。在这样的情况下,研究者们使用地质统计学方法来预测更大的区域(一维、二维和三维)空间上样本点的值。

克里金方法在地质统计学中有非常广泛的应用。正如刚才所提到的,克里金方法的提出就是在南非威特沃特斯兰德金矿的采矿工程师Krige在金矿开采方面的应用,用于估计当地金属浓度来获取更高的利润。在英法海底隧道建设时,Chilès就使用Kriging插值方法预测地层顶部和底部的深度,绘制泥灰岩的形态及其上下限,随着隧道的钻凿,工程师们发现地层上下限实际情况与预测精度相一致\textsuperscript{\cite{chiles2009geostatistics}}。

克里金方法不仅仅应用于地质统计学,在污染生态学、精准农业以及渔业学也都有比较多的应用。2014年中国科学院地理科学与资源研究所钟卜青、陶亮使用指标克里金法(IK)和多变量克里金法(MVIK)估算中国西南部喀斯特地区重金属超过有限混合物分布模型(FMDM)阈值的概率\textsuperscript{\cite{ZHONG2014422}};2012年A.Castrignanò利用共同克里金法和因子克里金法对橄榄种植区上橄榄果蝇种群密度的多变量空间(海拔)和时间(周期)数据进行分析,生成专题图描绘监测区域\textsuperscript{\cite{castrignano2012spatio}};2010年Charles F.Adams收集1999年至2007年在美国Nantucket岛附近的两个区域扇贝量,通过Kriging方法对物种密度进行估计,并绘制该区域内的扇贝分布数量,确保该区域的扇贝种群不会枯竭\textsuperscript{\cite{ADAMS2010460}}。