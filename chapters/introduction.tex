\section{介绍}
克里格法是一种地质统计学的预测方法,它是对准时支持或块支持的最佳线性无偏预测器;最好的是它的预测误差方差最小。在实践中,它是一种加权移动平均,其中的权重取决于变异函数和样本点在其目标附近的配置。到目前为止,普通克里格法是最流行的方法,部分原因是它在偏离基本假设方面是稳健的。然而,对于特定的任务,还有许多更高级的克里格方法。举例说明了改变变异函数参数和样本构型对克里格权值的影响、预测和预测方差。将准时克里格法和块克里格法进行比较,以映射预测和误差。在各向异性存在的克里格,简单克里格和对数正态克里格也得到了说明。给出了对数正态克里格对原尺度的反变换解。准时kriging可用于从“遗漏”交叉验证的诊断统计数据中识别合适的变异函数模型。