\section{区域化变量理论}
平稳性是克里金方法实用性的基础,假设插值区域数据视为有一定趋势的确定性变化,可以将插值空间场的数值分为两个部分,即确定性趋势函数值和随机自相关函数值:
\begin{equation}
    V \left( \mathbf{s} \right) = \mu \left( \mathbf{s} \right) + \varepsilon \left( \mathbf{s} \right)
    \label{区域化变量}
\end{equation}
其中,$ \mu \left( \mathbf{s} \right) $为确定性函数变量;$ \varepsilon \left( \mathbf{s} \right) $为均值为0,协方差为$ C\left( \mathbf{h} \right) $的随机自相关变量,其中$ \mathbf{h} $是空间中的相对距离,称为滞后。

协方差表达式为:
\begin{equation}
    C\left( \mathbf{h} \right) = E \left[ \varepsilon \left( \mathbf{s} \right) \varepsilon \left( \mathbf{s} + \mathbf{h} \right) \right]
    \label{协方差表达式}
\end{equation}

当$ \mu \left( \mathbf{s} \right) $为常数时(也就是普通克里金方法,$ \mu \left( \mathbf{s} \right) = \mu \left( \text{常数} \right) $时),协方差函数可以表达为:
\begin{equation}
    C\left( \mathbf{h} \right) = E \left[ \{ V\left( \mathbf{s} \right) - \mu \} \{ V\left( \mathbf{s} + \mathbf{h} \right) - \mu \} \right] = E \left[ \{ V \left( \mathbf{s} \right) \} \{ V \left( \mathbf{s} + \mathbf{h} \right) \} - \mu^{2} \right]
    \label{协方差函数}
\end{equation}
其中,$  V \left( \mathbf{s} \right) $和$ V \left( \mathbf{s} + \mathbf{h} \right) $表示空间场物理变量在$ \mathbf{s} $和$ \mathbf{s} + \mathbf{h} $的值,$ E $表示期望。

从公式\ref{协方差函数}看出,协方差仅与$ \mathbf{h} $有关,即样本之间在距离和方向上的间隔。这是基于二阶平稳性的假设,在实际空间场插值过程中,通常不能假设平均值是常数,否则方差就不存在。实际情况中,通常将平稳性假设改为Matheron所提出的内在平稳性假设\textsuperscript{\cite{matheron1965variables}},即期望差为0:
\begin{equation}
    E\left[ V\left( \mathbf{s} \right) - V \left( \mathbf{s} + \mathbf{h} \right) \right] = 0
\end{equation}

从而可以推导出变异函数与预测值方差的关系:
\begin{equation}
    Var \left[ V\left( \mathbf{s} \right) - V\left( \mathbf{s} + \mathbf{h} \right) \right] = E\left[ \{ V\left( \mathbf{s} \right) - V \left( \mathbf{s} + \mathbf{h} \right) \}^{2} \right] = 2 \gamma\left( \mathbf{h} \right)
\end{equation}
其中,$ \gamma\left( \mathbf{h} \right) $是滞后$ \mathbf{h} $的半方差,它是与相对位置$ \mathbf{h} $有关的变异函数。

对于二阶平稳状态,协方差函数和变异函数是等价的:
\begin{equation}
    \gamma \left( \mathbf{h} \right) = C \left( \mathbf{0} \right) - C \left( \mathbf{h} \right)
\end{equation}
其中,$ C \left( \mathbf{0} \right) = \sigma^{2} $表示方差。

因此,对于普通克里金模型,半方差函数可以看成:
\begin{equation}
    \begin{split}
        \gamma\left( \mathbf{h} \right)
        & = \frac{1}{2} E \left[ \{ \varepsilon\left( \mathbf{s} \right) - \varepsilon \left( \mathbf{s} + \mathbf{h} \right)\}^{2} \right] \\
        & = \frac{1}{2} E \left[ \{ V \left( \mathbf{s} \right) - V \left( \mathbf{s} + \mathbf{h} \right)\}^{2} \right]
    \end{split}
\end{equation}
